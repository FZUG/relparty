\documentclass{beamer}
% Author: Alick Zhao <alick9188@gmail.com>
% Author: Zamir SUN <zsun@fedoraproject.org>

% This file is modified from a solution template for:

% - Giving a talk on some subject.
% - The talk is between 15min and 45min long.
% - Style is ornate.

% Copyright 2004 by Till Tantau <tantau@users.sourceforge.net>.
%
% In principle, this file can be redistributed and/or modified under
% the terms of the GNU Public License, version 2.
%
% However, this file is supposed to be a template to be modified
% for your own needs. For this reason, if you use this file as a
% template and not specifically distribute it as part of a another
% package/program, I grant the extra permission to freely copy and
% modify this file as you see fit and even to delete this copyright
% notice.

\mode<presentation>
{
  \usetheme{Madrid}
  \setbeamercovered{transparent}
\definecolor{fedorablue}{RGB}{60,110,180}
\definecolor{fedoradarkblue}{RGB}{41,65,114}
\definecolor{fedoradarkgrey}{RGB}{76,76,76}
\setbeamercolor*{palette primary}{fg=white,bg=fedorablue}
\setbeamercolor*{palette secondary}{fg=white,bg=fedoradarkblue}
\setbeamercolor*{palette tertiary}{fg=white,bg=fedorablue}
\setbeamercolor*{palette quaternary}{fg=white,bg=black}
}

\usepackage{graphicx}
\graphicspath{{fig/}}
\usepackage{listings}
\usepackage{xspace}
\usepackage{fontspec}
\usepackage[CJKchecksingle]{xeCJK}

\usepackage{hyperxmp}
\hypersetup{
pdfauthor={Zamir SUN},
pdfcopyright={Copyright (C) 2014 by Zamir SUN.
Licensed under CC-BY-SA 4.0. Some rights reserved.
This work is based on Matthew Miller's previous work.
You need to follow the logo usage guideline to use the Fedora logo.},
pdflicenseurl={http://creativecommons.org/licenses/by-sa/4.0/},
}

% xeCJK conf setup
\punctstyle{kaiming}

\setCJKmainfont[BoldFont={WenQuanYi Micro Hei},
ItalicFont={AR PL UKai CN}]{AR PL UMing CN}
\setCJKsansfont{WenQuanYi Micro Hei}
\setCJKmonofont{WenQuanYi Micro Hei Mono}

\setCJKfamilyfont{zhsong}{AR PL UMing CN}
\setCJKfamilyfont{zhhei}{WenQuanYi Zen Hei}
\setCJKfamilyfont{zhkai}{AR PL UKai CN}

\newcommand*{\songti}{\CJKfamily{zhsong}} % 宋体
\newcommand*{\heiti}{\CJKfamily{zhhei}}   % 黑体
\newcommand*{\kaishu}{\CJKfamily{zhkai}}  % 楷书

\title{What's New in Fedora?}

\author[Zamir SUN] % (optional, use only with lots of authors)
{Zamir SUN\\ \texttt{zsun@fedoraproject.org}}

\date[Fedora 22 Release Party] % (optional)
{\today}

\subject{Fedora, Fedora.next}

% Delete this, if you do not want the table of contents to pop up at
% the beginning of each subsection:
\AtBeginSection[]
{
  \begin{frame}<beamer>{Outline}
    \tableofcontents[currentsection]
  \end{frame}
}

% If you wish to uncover everything in a step-wise fashion, uncomment
% the following command:

%\beamerdefaultoverlayspecification{<+->}

\hypersetup{
%pdfpagemode=FullScreen,
breaklinks=false,
}

\logo{\includegraphics[height=.15\textheight]{fedora_logo_cmyk.pdf}}

\begin{document}

\begin{frame}
  \titlepage
\end{frame}

\begin{frame}{Outline}
  \tableofcontents
  % You might wish to add the option [pausesections]
\end{frame}


% Since this a solution template for a generic talk, very little can
% be said about how it should be structured. However, the talk length
% of between 15min and 45min and the theme suggest that you stick to
% the following rules:

% - Exactly two or three sections (other than the summary).
% - At *most* three subsections per section.
% - Talk about 30s to 2min per frame. So there should be between about
%   15 and 30 frames, all told.

\section{Introduction}

\begin{frame}{About Me}
  \begin{itemize}
    \item Zamir SUN, aka zsun
    \item Linux user
    \item Fedora contributor
    \item Email: \texttt{zsun@fedoraproject.org}
  \end{itemize}
\end{frame}


\begin{frame}{It's an OS}
  \begin{itemize}
    \item GNU/Linux distro
      \begin{itemize}
        \item Community driven
        \item Sponsored by Red Hat
        \item Upstream to RHEL, CentOS
      \end{itemize}
      \item Supported Architectures
        \begin{itemize}
          \item x86, x86\_64
          \item ARM, ARM AArch 64
          \item s390x, PowerPC, MIPS
        \end{itemize}
  \end{itemize}
\end{frame}

\begin{frame}{Also a community}
  \begin{beamercolorbox}[center,ht=4ex,dp=0.5ex,rounded=true]{block body}
    {\centering The Fedora Project's mission is to lead the advancement of free and
  open source software and content as a collaborative community.}
  \end{beamercolorbox}
\end{frame}

\begin{frame}{Fedora Four Foundations}
  \begin{figure}[htbp]
    \centering
    \includegraphics[height=.7\textheight]{four_foundations.pdf}
  \end{figure}
\end{frame}

\section{What's new in Fedora 22}
% notable features available in Fedora 21
\begin{frame}{General}
      \begin{itemize}
        \item Kernel updated to 4.0
        \item Yum replaced by DNF
      \end{itemize}
    \end{frame}
\begin{frame}{Workstation}
      \begin{itemize}
        \item GDM on Wayland
        \item GNOME 3.16
        \item Message tray replaced by message list in GNOME Shell top bar
        \item KDE Plasma desktop 5
        \item Libinput used for input devices
      \end{itemize}
    \end{frame}
\begin{frame}{Server}
      \begin{itemize}
        \item XFS as default filesystem
        \item Kickstart syntax changed
        \item AArch64 QEMU/KVM VM Installation with libvirt and virt-manager Support
      \end{itemize}
    \end{frame}
\begin{frame}{Cloud}
    \begin{itemize}
      \item Image smaller than before
      \item RPM-OSTree by default, DNF and yum won't work by default
      \item New CI tool: Tunir
    \end{itemize}
\end{frame}


\begin{frame}{Better fit for Fedora.next}
  \begin{itemize}
    \item Additive
    \item Incremental improvement
    \item Planning, direction-setting
  \end{itemize}
\end{frame}

\section{Your involvement}

\begin{frame}{Groups}
  \begin{itemize}
    \item Sub-projects
      \begin{figure}[htbp]
        \centering
        \includegraphics[width=.9\textwidth]{subprojects.png}
      \end{figure}
  \end{itemize}
\end{frame}

\begin{frame}{You Can Contribute!}
  \begin{itemize}
    \item Translation, testing
    \item Design, coding, document, packaging, etc.
    \item Fedora Join SIG
    \item Join the discussion
      \begin{itemize}
        \item Mailing list
        \item IRC
      \end{itemize}
  \end{itemize}
\end{frame}


\section*{Summary}

\begin{frame}{Summary}
  % Keep the summary *very short*.
  \begin{itemize}
  \item Fedora: Freedom, Friends, Features, First
  \item Fedora 22: More stable, more agile
  \item Join Fedora: You can contribute!
  \end{itemize}
\end{frame}

\begin{frame}{Join us!}
  \begin{itemize}
    \item Mailing list: chinese@lists.fedoraproject.org
    \item Weibo: @fedoraproject \url{http://weibo.com/fedoraproject}
    \item irc: \#fedora-zh @ freenode.net
    \item fedoraproject.org: \url{http://fedoraproject.org/}
  \end{itemize}
\end{frame}


\end{document}
%%% vim: set sw=2 isk+=\: et tw=70 formatoptions+=mM:
